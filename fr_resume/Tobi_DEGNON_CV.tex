\documentclass[10pt, letterpaper]{article}

% Packages:
\usepackage[
    ignoreheadfoot, % set margins without considering header and footer
    top=2 cm, % seperation between body and page edge from the top
    bottom=2 cm, % seperation between body and page edge from the bottom
    left=2 cm, % seperation between body and page edge from the left
    right=2 cm, % seperation between body and page edge from the right
    footskip=1.0 cm, % seperation between body and footer
    % showframe % for debugging 
]{geometry} % for adjusting page geometry
\usepackage{titlesec} % for customizing section titles
\usepackage{tabularx} % for making tables with fixed width columns
\usepackage{array} % tabularx requires this
\usepackage[dvipsnames]{xcolor} % for coloring text
\definecolor{primaryColor}{RGB}{0, 0, 0} % define primary color
\usepackage{enumitem} % for customizing lists
\usepackage{fontawesome5} % for using icons
\usepackage{amsmath} % for math
\usepackage[
    pdftitle={Tobi DEGNON's CV},
    pdfauthor={Tobi DEGNON},
    pdfcreator={LaTeX with RenderCV},
    colorlinks=true,
    urlcolor=primaryColor
]{hyperref} % for links, metadata and bookmarks
\usepackage[pscoord]{eso-pic} % for floating text on the page
\usepackage{calc} % for calculating lengths
\usepackage{bookmark} % for bookmarks
\usepackage{lastpage} % for getting the total number of pages
\usepackage{changepage} % for one column entries (adjustwidth environment)
\usepackage{paracol} % for two and three column entries
\usepackage{ifthen} % for conditional statements
\usepackage{needspace} % for avoiding page brake right after the section title
\usepackage{iftex} % check if engine is pdflatex, xetex or luatex

% Ensure that generate pdf is machine readable/ATS parsable:
\ifPDFTeX
    \input{glyphtounicode}
    \pdfgentounicode=1
    \usepackage[T1]{fontenc}
    \usepackage[utf8]{inputenc}
    \usepackage{lmodern}
\fi

\usepackage{charter}

% Some settings:
\raggedright
\AtBeginEnvironment{adjustwidth}{\partopsep0pt} % remove space before adjustwidth environment
\pagestyle{empty} % no header or footer
\setcounter{secnumdepth}{0} % no section numbering
\setlength{\parindent}{0pt} % no indentation
\setlength{\topskip}{0pt} % no top skip
\setlength{\columnsep}{0.15cm} % set column seperation
\pagenumbering{gobble} % no page numbering

\titleformat{\section}{\needspace{4\baselineskip}\bfseries\large}{}{0pt}{}[\vspace{1pt}\titlerule]

\titlespacing{\section}{
    % left space:
    -1pt
}{
    % top space:
    0.3 cm
}{
    % bottom space:
    0.2 cm
} % section title spacing

\renewcommand\labelitemi{$\vcenter{\hbox{\small$\bullet$}}$} % custom bullet points
\newenvironment{highlights}{
    \begin{itemize}[
        topsep=0.10 cm,
        parsep=0.10 cm,
        partopsep=0pt,
        itemsep=0pt,
        leftmargin=0 cm + 10pt
    ]
}{
    \end{itemize}
} % new environment for highlights


\newenvironment{highlightsforbulletentries}{
    \begin{itemize}[
        topsep=0.10 cm,
        parsep=0.10 cm,
        partopsep=0pt,
        itemsep=0pt,
        leftmargin=10pt
    ]
}{
    \end{itemize}
} % new environment for highlights for bullet entries

\newenvironment{onecolentry}{
    \begin{adjustwidth}{
        0 cm + 0.00001 cm
    }{
        0 cm + 0.00001 cm
    }
}{
    \end{adjustwidth}
} % new environment for one column entries

\newenvironment{twocolentry}[2][]{
    \onecolentry
    \def\secondColumn{#2}
    \setcolumnwidth{\fill, 4.5 cm}
    \begin{paracol}{2}
}{
    \switchcolumn \raggedleft \secondColumn
    \end{paracol}
    \endonecolentry
} % new environment for two column entries

\newenvironment{threecolentry}[3][]{
    \onecolentry
    \def\thirdColumn{#3}
    \setcolumnwidth{, \fill, 4.5 cm}
    \begin{paracol}{3}
    {\raggedright #2} \switchcolumn
}{
    \switchcolumn \raggedleft \thirdColumn
    \end{paracol}
    \endonecolentry
} % new environment for three column entries

\newenvironment{header}{
    \setlength{\topsep}{0pt}\par\kern\topsep\centering\linespread{1.5}
}{
    \par\kern\topsep
} % new environment for the header

\newcommand{\placelastupdatedtext}{% \placetextbox{<horizontal pos>}{<vertical pos>}{<stuff>}
  \AddToShipoutPictureFG*{% Add <stuff> to current page foreground
    \put(
        \LenToUnit{\paperwidth-2 cm-0 cm+0.05cm},
        \LenToUnit{\paperheight-1.0 cm}
    ){\vtop{{\null}\makebox[0pt][c]{
        \small\color{gray}\textit{Last updated in November 2024}\hspace{\widthof{Last updated in November 2024}}
    }}}%
  }%
}%

% save the original href command in a new command:
\let\hrefWithoutArrow\href

% new command for external links:


\begin{document}
    \newcommand{\AND}{\unskip
        \cleaders\copy\ANDbox\hskip\wd\ANDbox
        \ignorespaces
    }
    \newsavebox\ANDbox
    \sbox\ANDbox{$|$}

    \begin{header}
        \fontsize{25 pt}{25 pt}\selectfont Tobi DEGNON

        \vspace{5 pt}

        \normalsize
        \mbox{Cotonou, BJ}%
        \kern 5.0 pt%
        \AND%
        \kern 5.0 pt%
        \mbox{\hrefWithoutArrow{mailto:tobidegnon@proton.me}{tobidegnon@proton.me}}%
        \kern 5.0 pt%
        \AND%
        \kern 5.0 pt%
        \mbox{\hrefWithoutArrow{https://oluwatobi.dev/}{oluwatobi.dev}}%
        \kern 5.0 pt%
        \AND%
        \kern 5.0 pt%
        \mbox{\hrefWithoutArrow{https://github.com/tobi-de}{github.com/tobi-de}}%
    \end{header}

    \vspace{5 pt - 0.3 cm}


    \section{Summary}



        
        \begin{onecolentry}
            Développeur backend Django avec 4 ans d'expérience dans des startups, où j'ai construit des produits de A à Z, de l'architecture au déploiement
        \end{onecolentry}


    
    \section{Experience}



        
        \begin{twocolentry}{
            Nov 2021 – present
        }
            \textbf{Développeur Web Full Stack}, BFTGROUP\end{twocolentry}

        \vspace{0.10 cm}
        \begin{onecolentry}
            \begin{highlights}
                \item Amélioration des performances d'une application Django à fort trafic utilisée par des millions d'utilisateurs, grâce à l'optimisation des requêtes et au cache, réduisant les temps de chargement de 80 \%
                \item Conception et mise en œuvre d'un outil de migration personnalisé pour transférer les données des clients depuis leur système existant vers notre plateforme, tout en garantissant l'intégrité des données et la continuité des activités
                \item Architecture, développement, test et documentation d'une application Django complète pour améliorer le suivi des contrats numériques pour un de nos clients
                \item Propositions régulières et développement d'outils pour améliorer l'expérience des développeurs, ce qui a conduit à une augmentation de la productivité et à la publication de plusieurs projets open source basés sur des idées réussies
                \item Développement d'un service Python responsable de la création d'environnements de test pour les développeurs afin de tester de nouvelles fonctionnalités
                \item Initiation et réalisation régulière de revues de code pour encourager le refactoring fréquent au sein de l'équipe
            \end{highlights}
        \end{onecolentry}


        \vspace{0.2 cm}

        \begin{twocolentry}{
            Feb 2020 – Oct 2021
        }
            \textbf{Développeur Web Full Stack (Télétravail)}, EDEV\end{twocolentry}

        \vspace{0.10 cm}
        \begin{onecolentry}
            \begin{highlights}
                \item Mise en place de toute l'infrastructure et des outils de l'entreprise, initialement déployés sur Heroku avant une migration vers un VPS pour réduire les coûts
                \item Architecture et développement de plusieurs applications Django complètes en tant que développeur unique
                \item Interaction directe avec les clients pour améliorer les solutions et corriger les bogues
                \item Intégration et mentorat des stagiaires
            \end{highlights}
        \end{onecolentry}



    
    \section{Projects}



        
        \begin{twocolentry}{
            Dec 2023 – present
        }
            \textbf{falco} -- falco.oluwatobi.dev\end{twocolentry}

        \vspace{0.10 cm}
        \begin{onecolentry}
            \begin{highlights}
                \item Une boîte à outils Django axée sur une expérience de développement plus rapide et meilleure. Falco est une compilation de mon expérience dans la création de meilleurs projets Django plus rapidement. Il inclut un modèle de projet, une génération de vues CRUD, une CI/CD intégrée, une gestion des versions de projet avec génération automatique de changelogs, et plus encore.
            \end{highlights}
        \end{onecolentry}


        \vspace{0.2 cm}

        \begin{twocolentry}{
            Oct 2024 – present
        }
            \textbf{fujin} -- fujin.oluwatobi.dev\end{twocolentry}

        \vspace{0.10 cm}
        \begin{onecolentry}
            \begin{highlights}
                \item Un outil de déploiement pour projets web. Initialement conçu comme un ensemble de scripts pour aider notre équipe à déployer rapidement des projets Django en production avec une excellente expérience de développement, il est maintenant emballé en tant qu'outil CLI open source.
            \end{highlights}
        \end{onecolentry}


        \vspace{0.2 cm}

        \begin{twocolentry}{
            Feb 2022 – present
        }
            \textbf{dj-shop-cart} -- tobi-de.github.io/dj-shop-cart\end{twocolentry}

        \vspace{0.10 cm}
        \begin{onecolentry}
            \begin{highlights}
                \item Conçu à l'origine comme un simple gestionnaire de panier d'achat, ce projet est devenu l'un des outils les plus fréquemment utilisés par notre équipe grâce à sa flexibilité.
            \end{highlights}
        \end{onecolentry}



    
    \section{Skills}



        
        \begin{onecolentry}
            \textbf{Backend:} Django, Python, Postgres, FastAPI
        \end{onecolentry}

        \vspace{0.2 cm}

        \begin{onecolentry}
            \textbf{Frontend:} HTML / CSS, Javascript, TailwindCSS, Bootstrap, HTMX, AlpineJS
        \end{onecolentry}

        \vspace{0.2 cm}

        \begin{onecolentry}
            \textbf{DevOps:} Linux, Docker, AWS
        \end{onecolentry}

        \vspace{0.2 cm}

        \begin{onecolentry}
            \textbf{Outils:} Git, Fabric, Nginx, Caddy, Sentry, PyCharm, VSCode
        \end{onecolentry}

        \vspace{0.2 cm}

        \begin{onecolentry}
            \textbf{En cours d'apprentissage:} Elixir, Phoenix
        \end{onecolentry}


    
    \section{Volunteer}



        
        \begin{twocolentry}{
            Feb 2023
        }
            \textbf{Open Source}\end{twocolentry}

        \vspace{0.10 cm}
        \begin{onecolentry}
            \begin{highlights}
                \item Contribution à divers projets open source de différentes manières, y compris l'implémentation de nouvelles fonctionnalités, le signalement de bogues, la mise à jour de la documentation, etc.
            \end{highlights}
        \end{onecolentry}



    
    \section{Education}



        
        \begin{twocolentry}{
            Nov 2016 – Jan 2020
        }
            \textbf{Institut De Formation Et De Recherche En Informatique (IFRI)}, Licence in Informatique (inachevé)\end{twocolentry}




    
    \section{Languages}



        
        \begin{onecolentry}
            \textbf{Anglais:} Professionnel
        \end{onecolentry}

        \vspace{0.2 cm}

        \begin{onecolentry}
            \textbf{Français:} Langue maternelle
        \end{onecolentry}


    
    \section{References}



        
        \begin{onecolentry}
            \textbf{Aimé An-Nyong DEGBEY, Responsable Développement BFT Group} -- linkedin.com/in/aimé-an-nyong-degbey\end{onecolentry}

        \vspace{0.10 cm}
        \begin{onecolentry}
            \begin{highlights}
                \item Tobi est un développeur Python très compétent et dévoué qui a apporté des contributions significatives à notre équipe et à nos projets chez BFT, livrant constamment un travail de haute qualité et démontrant d'excellentes compétences en résolution de problèmes.
            \end{highlights}
        \end{onecolentry}



    

\end{document}