\documentclass[10pt, letterpaper]{article}

% Packages:
\usepackage[
    ignoreheadfoot, % set margins without considering header and footer
    top=2 cm, % seperation between body and page edge from the top
    bottom=2 cm, % seperation between body and page edge from the bottom
    left=2 cm, % seperation between body and page edge from the left
    right=2 cm, % seperation between body and page edge from the right
    footskip=1.0 cm, % seperation between body and footer
    % showframe % for debugging 
]{geometry} % for adjusting page geometry
\usepackage{titlesec} % for customizing section titles
\usepackage{tabularx} % for making tables with fixed width columns
\usepackage{array} % tabularx requires this
\usepackage[dvipsnames]{xcolor} % for coloring text
\definecolor{primaryColor}{RGB}{0, 0, 0} % define primary color
\usepackage{enumitem} % for customizing lists
\usepackage{fontawesome5} % for using icons
\usepackage{amsmath} % for math
\usepackage[
    pdftitle={Tobi DEGNON's CV},
    pdfauthor={Tobi DEGNON},
    pdfcreator={LaTeX with RenderCV},
    colorlinks=true,
    urlcolor=primaryColor
]{hyperref} % for links, metadata and bookmarks
\usepackage[pscoord]{eso-pic} % for floating text on the page
\usepackage{calc} % for calculating lengths
\usepackage{bookmark} % for bookmarks
\usepackage{lastpage} % for getting the total number of pages
\usepackage{changepage} % for one column entries (adjustwidth environment)
\usepackage{paracol} % for two and three column entries
\usepackage{ifthen} % for conditional statements
\usepackage{needspace} % for avoiding page brake right after the section title
\usepackage{iftex} % check if engine is pdflatex, xetex or luatex

% Ensure that generate pdf is machine readable/ATS parsable:
\ifPDFTeX
    \input{glyphtounicode}
    \pdfgentounicode=1
    \usepackage[T1]{fontenc}
    \usepackage[utf8]{inputenc}
    \usepackage{lmodern}
\fi

\usepackage{charter}

% Some settings:
\raggedright
\AtBeginEnvironment{adjustwidth}{\partopsep0pt} % remove space before adjustwidth environment
\pagestyle{empty} % no header or footer
\setcounter{secnumdepth}{0} % no section numbering
\setlength{\parindent}{0pt} % no indentation
\setlength{\topskip}{0pt} % no top skip
\setlength{\columnsep}{0.15cm} % set column seperation
\pagenumbering{gobble} % no page numbering

\titleformat{\section}{\needspace{4\baselineskip}\bfseries\large}{}{0pt}{}[\vspace{1pt}\titlerule]

\titlespacing{\section}{
    % left space:
    -1pt
}{
    % top space:
    0.3 cm
}{
    % bottom space:
    0.2 cm
} % section title spacing

\renewcommand\labelitemi{$\vcenter{\hbox{\small$\bullet$}}$} % custom bullet points
\newenvironment{highlights}{
    \begin{itemize}[
        topsep=0.10 cm,
        parsep=0.10 cm,
        partopsep=0pt,
        itemsep=0pt,
        leftmargin=0 cm + 10pt
    ]
}{
    \end{itemize}
} % new environment for highlights


\newenvironment{highlightsforbulletentries}{
    \begin{itemize}[
        topsep=0.10 cm,
        parsep=0.10 cm,
        partopsep=0pt,
        itemsep=0pt,
        leftmargin=10pt
    ]
}{
    \end{itemize}
} % new environment for highlights for bullet entries

\newenvironment{onecolentry}{
    \begin{adjustwidth}{
        0 cm + 0.00001 cm
    }{
        0 cm + 0.00001 cm
    }
}{
    \end{adjustwidth}
} % new environment for one column entries

\newenvironment{twocolentry}[2][]{
    \onecolentry
    \def\secondColumn{#2}
    \setcolumnwidth{\fill, 4.5 cm}
    \begin{paracol}{2}
}{
    \switchcolumn \raggedleft \secondColumn
    \end{paracol}
    \endonecolentry
} % new environment for two column entries

\newenvironment{threecolentry}[3][]{
    \onecolentry
    \def\thirdColumn{#3}
    \setcolumnwidth{, \fill, 4.5 cm}
    \begin{paracol}{3}
    {\raggedright #2} \switchcolumn
}{
    \switchcolumn \raggedleft \thirdColumn
    \end{paracol}
    \endonecolentry
} % new environment for three column entries

\newenvironment{header}{
    \setlength{\topsep}{0pt}\par\kern\topsep\centering\linespread{1.5}
}{
    \par\kern\topsep
} % new environment for the header

\newcommand{\placelastupdatedtext}{% \placetextbox{<horizontal pos>}{<vertical pos>}{<stuff>}
  \AddToShipoutPictureFG*{% Add <stuff> to current page foreground
    \put(
        \LenToUnit{\paperwidth-2 cm-0 cm+0.05cm},
        \LenToUnit{\paperheight-1.0 cm}
    ){\vtop{{\null}\makebox[0pt][c]{
        \small\color{gray}\textit{Last updated in November 2024}\hspace{\widthof{Last updated in November 2024}}
    }}}%
  }%
}%

% save the original href command in a new command:
\let\hrefWithoutArrow\href

% new command for external links:


\begin{document}
    \newcommand{\AND}{\unskip
        \cleaders\copy\ANDbox\hskip\wd\ANDbox
        \ignorespaces
    }
    \newsavebox\ANDbox
    \sbox\ANDbox{$|$}

    \begin{header}
        \fontsize{25 pt}{25 pt}\selectfont Tobi DEGNON

        \vspace{5 pt}

        \normalsize
        \mbox{Cotonou, BJ}%
        \kern 5.0 pt%
        \AND%
        \kern 5.0 pt%
        \mbox{\hrefWithoutArrow{mailto:tobidegnon@proton.me}{tobidegnon@proton.me}}%
        \kern 5.0 pt%
        \AND%
        \kern 5.0 pt%
        \mbox{\hrefWithoutArrow{https://oluwatobi.dev/}{oluwatobi.dev}}%
        \kern 5.0 pt%
        \AND%
        \kern 5.0 pt%
        \mbox{\hrefWithoutArrow{https://github.com/tobi-de}{github.com/tobi-de}}%
    \end{header}

    \vspace{5 pt - 0.3 cm}


    \section{Summary}



        
        \begin{onecolentry}
            Backend Django developer with 4 years of experience at startups, where I built products from the ground up, from architecture to deployment
        \end{onecolentry}


    
    \section{Experience}



        
        \begin{twocolentry}{
            Nov 2021 – present
        }
            \textbf{Full Stack Web Developer}, BFTGROUP\end{twocolentry}

        \vspace{0.10 cm}
        \begin{onecolentry}
            \begin{highlights}
                \item Improved performance of a high-traffic Django application serving millions of users through query optimization and caching, reducing load times by 80\%
                \item Designed and implemented a custom migration tool to successfully transition client data from their legacy system to our platform, ensuring data integrity and business continuity
                \item Architected, developed, tested and documented a full-stack Django application to improve digital contract tracking for one of our customers
                \item Regularly suggested and built tools for my team to improve the developer experience, which led to increased productivity and the release of several open-source projects based on successful ideas
                \item Developed a Python service responsible for setting up staging environments for developers to test new features
                \item Regularly initiated and conducted code reviews to encourage frequent code refactoring within the team
            \end{highlights}
        \end{onecolentry}


        \vspace{0.2 cm}

        \begin{twocolentry}{
            Feb 2020 – Oct 2021
        }
            \textbf{Full Stack Web Developer (Remote)}, EDEV\end{twocolentry}

        \vspace{0.10 cm}
        \begin{onecolentry}
            \begin{highlights}
                \item Set up the entire infrastructure and tooling of the company, initially deployed on Heroku before migrating to a VPS to reduce costs
                \item Architected and developed multiple full-stack Django apps from scratch as a one-person team
                \item Interacted directly with customers to improve solutions and fix bugs
                \item Onboarded and mentored trainees
            \end{highlights}
        \end{onecolentry}



    
    \section{Projects}



        
        \begin{twocolentry}{
            Dec 2023 – present
        }
            \textbf{falco} -- falco.oluwatobi.dev\end{twocolentry}

        \vspace{0.10 cm}
        \begin{onecolentry}
            \begin{highlights}
                \item A Django toolbox focused on faster and better developer experience. Falco is a compilation of my experience in building better Django projects faster. It comes with a project template starter, CRUD views generation, baked-in CI/CD pipeline, project version management with auto changelog generation, and more.
            \end{highlights}
        \end{onecolentry}


        \vspace{0.2 cm}

        \begin{twocolentry}{
            Oct 2024 – present
        }
            \textbf{fujin} -- fujin.oluwatobi.dev\end{twocolentry}

        \vspace{0.10 cm}
        \begin{onecolentry}
            \begin{highlights}
                \item A deployment tool for web projects. Started as a set of scripts to help our team quickly deploy Django projects to production with a nice developer experience, and has now been packaged into an open-source CLI tool
            \end{highlights}
        \end{onecolentry}


        \vspace{0.2 cm}

        \begin{twocolentry}{
            Feb 2022 – present
        }
            \textbf{dj-shop-cart} -- tobi-de.github.io/dj-shop-cart\end{twocolentry}

        \vspace{0.10 cm}
        \begin{onecolentry}
            \begin{highlights}
                \item Originally designed as a simple shopping cart manager, this project evolved into one of our team's most frequently used utilities due to its flexibility
            \end{highlights}
        \end{onecolentry}



    
    \section{Skills}



        
        \begin{onecolentry}
            \textbf{Backend:} Django, Python, Postgres, FastAPI
        \end{onecolentry}

        \vspace{0.2 cm}

        \begin{onecolentry}
            \textbf{Frontend:} HTML / CSS, Javascript, TailwindCSS, Bootstrap, HTMX, AlpineJS
        \end{onecolentry}

        \vspace{0.2 cm}

        \begin{onecolentry}
            \textbf{DevOps:} Linux, Docker, AWS
        \end{onecolentry}

        \vspace{0.2 cm}

        \begin{onecolentry}
            \textbf{Tools:} Git, Fabric, Nginx, Caddy, Sentry, PyCharm, VSCode
        \end{onecolentry}

        \vspace{0.2 cm}

        \begin{onecolentry}
            \textbf{Currently Learning:} Elixir, Phoenix
        \end{onecolentry}


    
    \section{Volunteer}



        
        \begin{twocolentry}{
            Feb 2023
        }
            \textbf{Open Source}\end{twocolentry}

        \vspace{0.10 cm}
        \begin{onecolentry}
            \begin{highlights}
                \item Contributing to various open-source projects in multiple ways, including implementing new features, filing bug reports, updating documentation, etc.
            \end{highlights}
        \end{onecolentry}



    
    \section{Education}



        
        \begin{twocolentry}{
            Nov 2016 – Jan 2020
        }
            \textbf{Institut De Formation Et De Recherche En Informatique (IFRI)}, Bachelor in Computer Science (incomplete)\end{twocolentry}




    
    \section{Languages}



        
        \begin{onecolentry}
            \textbf{English:} Professional Working
        \end{onecolentry}

        \vspace{0.2 cm}

        \begin{onecolentry}
            \textbf{French:} Native Speaker
        \end{onecolentry}


    
    \section{References}



        
        \begin{onecolentry}
            \textbf{Aimé An-Nyong DEGBEY, Head of Development BFT Group:} 
        \end{onecolentry}


    

\end{document}